\documentclass[12pt]{article}

\usepackage{graphicx}
\usepackage{paralist}
\usepackage{listings}
\usepackage{booktabs}

\oddsidemargin 0mm
\evensidemargin 0mm
\textwidth 160mm
\textheight 200mm

\pagestyle {plain}
\pagenumbering{arabic}

\newcounter{stepnum}

\title{Assignment 1 Solution}
\author{Michael Barreiros barreirm}
\date{\today}

\begin {document}

\maketitle

The purpose of the program created in Assignment 1 was to read and process data from first-year engineering students
and to allocate them into one of their three choices if possible.

\section{Testing of the Original Program}

The rationale for my test cases for the module CalcModule was that all problems that might have occurred during reading data were dealt
with in the ReadAllocationData module. 

Some problems could have been that:
\begin{itemize}

\item GPAs are out of the 0.0-12.0 range
\item Choices were not of the choices available
\item Any issues with capitalization 
\item Any mismatch of data types or invalid input

\end{itemize}

With this in mind my CalcModule passed all my test cases. My testCalc Module had 10 test cases.

\subsection {Testing for Average}

This includes 4 tests. Testing the average of a student list for each gender. Then testing the average of a student list where no member of 
the gender exists, again this was done for each gender. An assumption was made when testing for the average of a gender with no members that 
the average function would return 0 and not None.

\subsection {Testing for Sort}

This includes 4 tests. Testing for the sorting of students all with unique GPAs. Testing for students with some of the students sharing the same 
GPA. It is assumed that students with the same GPA get sorted in the way they appear in the original student list. Next, I tested the sorting on
an empty list, the expected result was an empty list. The last test for Sort was testing an already sorted list. The expected result is to have 
the same list returned.

\subsection {Testing for Allocate}

This includes 2 tests. The first test really was testing many things. The first test of allocation tested for the allocation of a free choice
student that has a GPA lower than someone without free choice, it is expected that the free choice student gets allocated first.
The next test here tested the allocation of a student with a 4.0 GPA, I made the assumption that any student with a GPA lower than or equal to 
a 4.0 does not get allocated. The next test here tested to see what would happen when students apply for a department with not enough space to 
accommodate everyone. I made the assumption that even if a department goes over capacity with all free choice students the department cannot 
allocate more students than their initial capacity. Therefore in this test, we expect any student applying for a department who has reached their 
capacity to move on to their next choice. The final test here was to test what happens when a free choice student has a GPA below 4.0. As stated
earlier before we expect all students with a GPA at or below a 4.0 to not get allocated, this includes students with free choice.
The second test was a test for the allocation of zero students, the expected result was a dictionary with departments as keys and an empty list
as each of the key's value.

\section{Results of Testing Partner's Code}

\subsection{First Attempt}

I was unable to test my partner's code because he implemented two modules named al\_constants.py and al\_utility.py.
My partner used al\_contants to implement MIN\_GPA, MAX\_GPA, MIN\_PASSING\_GPA, GENDERS functions.
My partner used al\_utility to implement the remove\_duplicates.


\subsection{Second Attempt}
Steps taken to get the program to run:
\begin{enumerate}

\item Commented out the two lines of import statements
\item Made variables for MIN\_GPA, MAX\_GPA, MIN\_PASSING\_GPA with values 0.0, 12.0, and 4.0 respectively
\item Commented out remove\_duplicates(students) from the function allocate(S, F, C)

\end{enumerate}

The program ran, three of my tests failed. 

The tests that failed include: 
\begin{enumerate}

\item Testing for male average when there exists no males
\item Testing for female average when there exists no females
\item Testing allcation of students. The particular part that failed was the allocation of a student with a 4.0 GPA 

\end{enumerate}

\section{Discussion of Test Results}

\subsection{Problems with Original Code}

Through my test cases, my code passed every one of them. I cannot say that there are zero problems based on my code because a lot of the 
reasoning behind the decisions made on how to code something or what to do in a certain scenario came from assumptions that I had to make.

I would say that the majority of problems arise from ReadAllocationData and the assumptions I had to make when creating the needed
functions. 

For readStdnts, the assumption was made that in the file every student appeared in a new line and had entries for "macid 
fname lname gender GPA choice1 choice2 choice3" only appearing in that exact order with spaces between each entry and did not include the
key's needed for their dictionary counterparts. This meant that I did not expect "madId: barreirm" I expected only "barreirm". When 
creating the function I also accounted for some special characters like "\{", "\}", """, ",", "[", and "]". 

For readFreeChoice, the assumption that was made was that only students with free choice appear in the file and that the student's macId is
in the file. They could all be on one line but must be separated by at least one space, they could also all be on separate lines. 

For readDeptCapacity the assumption that was made was that in the input file the department's name comes first then the capacity for that
department separated by at least one space. Data entries could all be on separate lines or on one line or a mixture of the two options. 
This function accounted for some special characters like ":" and tabbing. If I were to return to this function I would have accounted for 
"\{" and "\}" just in case the incoming file tried to structure their data like a dictionary already.


\subsection{Problems with Partner's Code}

My partner's CalcModule failed three of my tests. The reason for failing the first two items on the list of failed tests was that 
my partner returned None rather than 0 when there were no members of the gender that you were trying to find the average for. This wasn't 
a problem with the code just a difference in assumptions.

The third item on the list of failed tests occurred because my partner assumed that students that have a 4.0 GPA still get allocated while I
made the assumption that they do not. Again, this was not a problem with their code, just a difference of assumptions. I made the assumption 
based on the design specifications for the function allocate(S,F,C) where it states "The algorithm for the allocation will allocate all 
students with a GPA greater than 4.0." I interpreted that as meaning anyone that doesn't have a GPA greater than 4.0, this includes 4.0, will
not get allocated.

\section{Critique of Design Specification}

One general critique of the design specifications is that it was very ambiguous and it required the designer to make a lot of assumptions. 
If this was a one person project then it might not be a big deal but if it were a project where many people were working on it and their 
individual modules or programs had to work together then all ambiguities must be dealt with in the design specification. Through testing my
own program and testing someone else's program I can see that we made different assumptions in some cases which resulted in different outputs
in some scenarios. 

I think that the for ReadAllocationData the design specification should have told us exactly what to expect as input when it comes to the
structure of the file and the data within each file. This would allow us to deal with the incoming data appropriately. This would make the 
process of creating this method less tedious because we wouldn't have to think about all possible scenarios of incoming data and how we would
deal with it if ever encountered. 

For CalcModule the design specification should have made it more clear what to do when encountering certain scenarios such as what to do with
a student who has a GPA of exactly 4.0. Another thing that the design specification should have told us is what happens when free choice
students all apply to a certain department and go over its capacity. It should tell us whether or not the department would allow for its 
capacity to increase since a promise was made to free choice students that they will get their first choice. Another thing that should be
specified is what to return when computing the average when there is no population for that gender, is it 0? Is it None? Another ambiguity
is what happens to students where all three choices are filled. Do they get allocated to a random department that has space? If so how does the
randomization of departments occur? Do they get a list of open departments to make a new list of choices and then start the allocation process 
again with other students who did not get allocated?
%\newpage

\section{Answers to Questions}

\begin{enumerate}[(a)]

\item I would make average(L,g) more general by allowing for the computation of average to occur over different filters. Meaning I would allow
for the computation of average for a certain department or based on student's first choice. I would also allow to compute the average of students
who's first name begin with a certain letter. The same could be said for their macid's and their last name. As for sort I would allow for the
option of sorting in increasing or decreasing order. I would also include the ability to sort by more than just by GPA. I would allow for the
ability to sort be alphabetical order for all options of macid, first name, or last name. I would also allow for the ability to sort by the first 
choice. With all of this I would also allow for the combination of different kinds of sorting. Such as sorting first by the department that is 
their first choice then sort by GPA. 

\item Aliasing in this context means having different variables referring to the same dictionary. The dictionary referred to by the two variables
is said to have two aliases. Since dictionaries are mutable data structures this can be a concern. If you try to copy a dictionary by simply
creating a new dictionary this may cause problems. If you change the values of keys or add/delete keys in this new "copy" it will affect the
original dictionary that you tried to copy as well. In order to guard against this one would use deepcopy in order to create a completely 
independent copy of the dictionary. 

\item A potential test case would have been reading students for readStdnts who's data weren't in the right location, such as their GPA being where
fname should be or if some entries were empty, such as not having a third choice. A potential test case for all three functions would have been to 
test for garbage input. This could include random characters or incorrect department names. A potential test case for all three functions could have
been to read from an empty file. Another test could have been to try to read from a file that does not exist. I believe CalcModule.py was selected
over ReadAllocationData.py as the one we should test because there were fewer variations and assumptions that would have to be made when 
creating the CalcModule module. Also, a lot of the "user" input errors that would have to be considered when testing ReadAllocationData don't exist when 
testing for CalcModule since the assumption is made that all "user" input errors were taken care of in ReadAllocationData.

\item The problems with using strings in this way is that the key or the value would have to be inputted exactly as it is shown in the dictionary
and would allow for zero variations such as a capital letter in the wrong position. A better approach would be to use named tuples like Professor 
Smith suggested in class. This allows us to create a set of key values to be used for the keys of our dictionary and in the 'male', 'female' example
this allows us to create a set of expected values for gender and we can access that set by having the fieldname of the set be gender.

\item The mathematic definition of a tuple is a collection of elements of possibly different types. Each tuple has one or more fields and each field
has a unique identifier called the field name. By this definition, dictionaries, named tuple, and a custom class all can be used to implement the 
mathematical notion of tuples in python. I would recommend changing students to namedtuple instead of a dictionary. Each student's choices can be 
implemented using a namedtuple. I think that changing to namedtuples is a good idea because the students and their choices will not change after 
creation so that the fact that tuples are immutable.

\item If the list of strings was changed to a different data structure, like a tuple, then CalcModule wouldn't have to be modified as I iterated over 
the entries of the list. You iterate over the entries of a tuple in the same way. I have never worked with tuples but that is my understanding. I do
not believe that a custom class would have to be modified if its data structure changed. The way I coded my scan through the students' choices was by
running a for loop that iterated over the first three indexes of the choice's list. If a custom class were to return true when trying to access a choice
that is not there then it would not change anything since I would never be reaching that point.

\end{enumerate}

\newpage

\lstset{language=Python, basicstyle=\tiny, breaklines=true, showspaces=false,
  showstringspaces=false, breakatwhitespace=true}
%\lstset{language=C,linewidth=.94\textwidth,xleftmargin=1.1cm}

\def\thesection{\Alph{section}}

\section{Code for ReadAllocationData.py}

\noindent \lstinputlisting{../src/ReadAllocationData.py}

\newpage

\section{Code for CalcModule.py}

\noindent \lstinputlisting{../src/CalcModule.py}

\newpage

\section{Code for testCalc.py}

\noindent \lstinputlisting{../src/testCalc.py}

\newpage

\section{Code for Partner's CalcModule.py}

\noindent \lstinputlisting{../partner/CalcModule.py}

\newpage

\section{Makefile}

\lstset{language=make}
\noindent \lstinputlisting{../Makefile}

\end {document}
